\chapter{Conclusion}

As a solution to the enclosing threat of rising shorelines, we have developed a proof of concept system called EcoBeach. EcoBeach is a highly scalable system that monitors water content changes from satellite imagery in real-time. EcoBeach's data pipeline ensures high availability and few points of failure. Data is continuously fed to a distributed MongoDB database, so it is readily available from an Android application that can query, plot, and visualize the monitored locations in correlation to a user's location.

The data pipeline resides in a docker swarm cluster that consists of three servers, two in Germany - more specifically in Nuremberg and Falkenstein. The third server is in Helsinki, Finland, which is the name node of our cluster. Each server has a scraper that downloads and processes satellite imagery from Copernicus. The scrapers publish their data to a specific Kafka topic for each node. Apache Spark master also runs on the Helsinki node, distributing spark jobs to worker nodes to analyze the beach images. 
Once data has been analyzed, it is saved to another Kafka topic. A Kafka Connect Sink is set up to feed the data from this topic into our distributed MongoDB database. The database is hosted on two nodes for redundancy.

Our frontend is an Android application that runs the Google Maps API to display a map the user can navigate. Once the application is loaded, it requests data from one of the Web API endpoints that also resides in the cluster. The API then makes a data request to the database and maps the received data into one or more beach objects. The API then sends the object in a JSON format to the app, which processes it and uses the geolocation data to place each beach on the map.

The system's goal is to provide users a simple way to see how their environment changes in real-time due to climate change and raise awareness of how the sea level rises around the world. 