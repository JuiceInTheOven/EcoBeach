\chapter{The Problem} \label{ch:the-problem}

In this chapter we will describe the problem that is addressed by the system and application built for the Semester Project in Scalable Systems, on the first semester on the masters of Software Engineering SDU. \medbreak
\noindent
First a problem definition will be given, along with the overall objective. Next an in depth problem description is presented, where the sub problems will be unveiled, and why it is necessary to derive a solution.

\section{Problem and Objective}

In present time the ecology is threatened by increasing changes to the climate. One of the notable climate changes is the rising shorelines that are predicted to rise exponentially in this century \todo{find source}. To combat this it is extremely important to react, and try to minimize climate change, but sadly this might not be something humanity is able to do in a timely manner. Therefore being able to monitor how the shorelines are changing can be paramount to alleviate the risk of rising shorelines. This prefaces the problem that is: \medbreak 
\noindent
Humanity might not be able to combat climate change in a timely manner to avoid massive floods in cities and countries, and currently there is not enough options to monitor how shorelines are changing to be able to prepare for or predict floods. \medbreak 
\noindent
This project aims to solve this problem by creating a system capable of processing satellite imagery of geo locations around the world in real-time to determine how the shorelines have changed and are changing. \medbreak 
\noindent
To feasible create such a system the project will rely on various big data technologies and tools, as well as the use of mobile sensing.

\section{Problem Description} \label{sec:problem-description}

Monitoring shorelines is quite interesting; having historical and current data on shoreline changes can potentially help both the public and the private sector.  \medbreak 
\noindent
Soon governments might need to build dams, or barriers to combat the rising shorelines, and avoid floods. Knowing where shorelines are rising the most, can be a huge factor in avoiding large floodings and thus the destruction of properties and potentially cities.

Some governments already have had the need for this which is apparent when taking a look at the Netherlands that have already built very resilient solutions to prevent floods e.g. the Maeslantkering storm surge barrier. \todo{add reference to Maselantkering}  \medbreak 
\noindent
In the private sector it might also be really helpful to know how shorelines are changing. This can be used to make informed decisions about where to settle down, or to allow preparation for floods for individuals living at places at risk.  \medbreak 
\noindent
From a technical perspective creating a system capable of processing satellite imagery in real-time is a daunting task, that has many sub-problems that needs to be solved to make it feasible.
To derive a solution we believe the following sub-problems must be solved:

\begin{itemize}
    \item How to download satellite images from Cupernicus?
    \item How to process images so water is differentiated from land?
    \item How to build a system capable of handling huge amounts of data? 
    \item How to build a system capable of real-time processing?
    \item How to build a highly resilient system?
    \item How to build a mobile application that uses mobile sensing in a meaningful way to visualize shoreline changes?
\end{itemize}
\noindent
It is paramount that these problems can be solved to derive a feasible solution to monitor shorelines in real-time. Such a solution, EcoBeach, is described in detail in the next chapter.
