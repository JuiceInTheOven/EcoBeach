\chapter{The Problem} \label{ch:the-problem}

\todo{For the full section: Add data on how the future looks in regard to flooding --> if data supports it rewrite places where we show uncertainty in flooding becoming a reality. If we know it we might as well say so.}
In this chapter, we will describe the problem addressed by EcoBeach, a system built as part of the Semester Project in Scalable Systems, on the first semester on the masters of Software Engineering SDU. \medbreak
\noindent
First, a problem definition will be given, along with the overall objective. Next, an in-depth problem description is presented, where the sub-problems will be unveiled, and why it is necessary to derive a solution.

\section{Problem and Objective}

At present, the ecology is threatened by increasing changes to the climate. One of the notable climate changes is the rising shorelines that are predicted to rise exponentially in this century \todo{find source}. It is imperative to react and try to minimize climate change to combat this, but sadly, this might not be something humanity can do on time. Therefore, monitoring how the shorelines are changing can be paramount to alleviate the risk of rising shorelines. This prefaces the problem that is: \medbreak 
\noindent
Humanity might not combat climate change in time to avoid massive floods in cities and countries, and currently there are not enough accessible options to monitor how shorelines are changing to be able to prepare for or predict floods. \medbreak 
\noindent
This project aims to solve this problem by creating a system capable of processing satellite imagery of geo-locations worldwide in real-time to determine how the shorelines have changed and are changing. \medbreak 
\noindent
To feasible create such a system, the project will rely on various big data technologies and tools and mobile sensing.

\section{Problem Description} \label{sec:problem-description}

Monitoring shorelines is quite interesting; having historical and current data on shoreline changes can potentially help both the public and the private sector.  \medbreak 
\noindent
Soon governments might need to build dams or barriers to combat the rising shorelines and avoid floods. Knowing where shorelines are rising the most can be a huge factor in preventing large floodings and destroying properties and, potentially, cities.

Some governments already have required this, which is apparent when looking at the Netherlands that have built resilient solutions to prevent floods, e.g., the Maeslantkering storm surge barrier. \todo{add reference to Maselantkering}  \medbreak 
\noindent
It might also be beneficial to know how shorelines are changing in the private sector. Knowing this can be helpful to make informed decisions about where to settle down or prepare for floods for individuals living at places at risk.  \medbreak 
\noindent
From a technical perspective, creating a system capable of processing satellite imagery in real-time is a daunting task with many sub-problems that need to be solved to make it feasible.
To derive a solution, we believe the following sub-problems must be solved:

\begin{itemize}
    \item How to download satellite images from Cupernicus?
    \item How to process images so water is differentiated from land?
    \item How to build a system capable of handling huge amounts of data? 
    \item How to build a system capable of real-time processing?
    \item How to build a highly resilient system?
    \item How to build a mobile application that uses mobile sensing in a meaningful way to visualize shoreline changes?
\end{itemize}
\noindent
It is paramount that these problems can be solved to derive a feasible solution to monitor shorelines in real-time. Such a solution, EcoBeach, is described in detail in the next chapter.
